\chapter{Literature Review}
\label{chap:lr}
\chaptermark{Literature Review}

\section{sEMG-Based Hand Pose Estimation}

Surface electromyography (sEMG) is widely used to estimate hand gestures and joint angles by measuring electrical activity from forearm muscles \cite{oskoei2007myoelectric, simao2019review}. Traditional approaches often use handcrafted features (e.g., mean absolute value, waveform length, or zero crossings) followed by classifiers such as support vector machines (SVM) or multilayer perceptrons (MLP) \cite{oladazimi2012review, liu2007recognition}. However, these methods typically lack robustness to motion complexity and signal variability \cite{parajuli2019real}. Moreover, the development of effective handcrafted features often requires domain-specific knowledge and extensive manual tuning by qualified specialists, which inherently limits generalizability and scalability across different tasks and user populations \cite{atzori2016deep, oskoei2008support, phinyomark2018feature}.

Recent work has focused on deep learning models, such as convolutional neural networks (CNNs), recurrent neural networks (RNNs), and temporal convolutional networks (TCNs) \cite{ameri2019regression, briouza2021convolutional, zhang2023lstm}, which can automatically extract and model temporal features from raw or minimally processed sEMG signals. These models show improved accuracy over traditional pipelines but still rely on fixed temporal windows and often ignore inter-channel dependencies \cite{lee2022explainable}.

In response, several researchers have explored attention-based and adaptive models, enabling selective focus on more informative regions of the signal \cite{yang2025stcnet, hu2019semg}. Yet, few works explicitly address the simultaneous adaptation in both time and channel space. Our proposed approach, termed \textit{Spatiotemporal Sampling}, aims to fill this gap by learning a sampling window and spatial activation pattern jointly from the data.

\section{Multimodal sEMG Datasets}

A number of public datasets have been developed for training and benchmarking sEMG-based models. The \textbf{NinaPro} dataset family is among the most widely used, offering synchronized sEMG and kinematic glove data for a variety of gestures \cite{zia2018multiday}. While valuable, NinaPro focuses on short, predefined movements and generalization across subjects, which makes it less suited for intra-subject studies involving longer, continuous sessions.

The \textbf{EMG2Pose} dataset introduced a large-scale, high-resolution multimodal dataset for fine-grained pose regression \cite{salter2024emg2pose}. It includes synchronized 3D motion capture and EMG recordings, offering unprecedented quality. However, the system setup involves 26 professional OptiTrack cameras and high-end Delsys sensors, resulting in a prohibitively expensive configuration for most academic labs.

At the other end of the spectrum, several low-cost or DIY setups exist, often based on consumer-grade devices like the Myo armband or custom EMG acquisition boards (e.g., based on Arduino or ADS1299 ICs) \cite{nasri2020semg}. These systems typically suffer from limitations such as:
\begin{itemize}
    \item Insufficient sampling rates (e.g., below 12bit 2\,kHz),
    \item Lack of synchronized, continuous hand tracking at 30\,fps or higher \cite{graf2023combining},
\end{itemize}

\section{Deep Learning Models for EMG-to-Pose Estimation}

The EMG2Pose dataset is complemented by the introduction of a deep learning model called \textbf{VEMG2Pose}, which takes a dynamic perspective on pose prediction \cite{salter2024emg2pose}. Rather than estimating static joint angles, it predicts joint angular velocities from EMG and integrates them over time to reconstruct the pose.

\textbf{NeuroPose} takes a different approach, focusing on direct estimation of joint angles rather than velocities. Its architecture is based on a U-Net with residual bottlenecks \cite{lee2022explainable}.

\textbf{SensingDynamics}, on the other hand, replaces U-Net with a more streamlined architecture. It includes learnable muscle unit activations and combines low-pass filtered EMG input with raw data to boost performance under noisy conditions \cite{zanghieri2023semg}.

These models represent state-of-the-art efforts to map EMG to pose, but they still rely on predefined featurization and structured down-sampling schemes. Our work builds upon this foundation by introducing a more flexible spatiotemporal selection mechanism — allowing the model to learn which time segments and channels are most informative on a per-instance basis.

\section{Research Gap}

From the review above, two major gaps emerge:
\begin{enumerate}
    \item \textbf{Modeling:} Few works jointly learn where (in time) and what (in channel space) to attend to in sEMG signals. Most deep models use fixed-length windows and treat all channels equally. The concept of learned \textit{spatiotemporal sampling} remains largely unexplored in sEMG modeling.
    \item \textbf{Data Acquisition:} There is no publicly available, cost-effective dataset acquisition system that provides both high-frequency sEMG data and synchronized 3D hand tracking at sufficient frame rates \cite{seo2024posture}.
\end{enumerate}

This thesis addresses both gaps: by proposing a model that learns to extract informative spatiotemporal segments from EMG, and by building a custom synchronized EMG+vision capture system that balances affordability and data fidelity.