\chapter{Conclusion}
\label{chap:conclusion}
\chaptermark{Conclusion}

\section{Summary of Contributions}

\begin{enumerate}[label=\arabic*.]

\item \textbf{Open, Low-Cost Capture Pipeline}.
      We demonstrated that a four-webcam array plus a \$40 Seeeduino-based
      sEMG board can deliver sub-millimetre pose reconstruction at
      \SI{32}{fps} and \SI{2.048}{kHz} raw EMG.
      The entire system -- hardware files, firmware, Python capture stack, and a
      storage format \(40\times\) smaller than EMG2Pose HDF5 -- is released
      under MIT and free-to-use terms.

\item \textbf{Adaptive Spatiotemporal Featuriser}.
      STS learns \emph{where} and \emph{when} to sample the EMG, inserting the
      result into a skip-connected MLP that also sees pose history, velocity
      and acceleration. On high-quality recordings STS equals or exceeds a
      carefully tuned CNN; on noisy or highly segmented data its advantage
      diminishes, confirming RQ-B.

\item \textbf{Hyper-parameter Study Across Datasets}.
      A unified Bayesian+Hyperband sweep (millions of theoretical combinations)
      shows that most hyper-parameters are architecture-agnostic, but some are
      strongly model-specific: CNNs crave larger kernels and lower L\(_1\),
      whereas STS cares about slice granularity and benefits from stronger
      \(\ell_1\) sparsity.

\item \textbf{Evidence for Motion Predictability}.
      Experiments with the EMG channel muted reveal only modest degradation on
      EMG2Pose and a larger drop on our dataset, implying that long, random,
      unscripted sessions indeed prevent a network from relying solely on
      kinematic extrapolation (answering RQ-C).

\end{enumerate}

\section*{Answers to the Research Questions}

\begin{description}[style=unboxed,leftmargin=0pt]

\item[\textbf{RQ-A}]
      The budget hardware is adequate for optical tracking but \emph{marginal}
      for sEMG: landmark error bottoms out at \(\sim\)22 mm on our board vs
      17 mm on EMG2Pose`s high-end sensors. Upgrading the analogue front-end
      would likely pay larger dividends than further model tweaks.

\item[\textbf{RQ-B}]
      STS outperforms CNN by \(\approx\)6 \% on clean data and ties it on noisy
      data; thus the adaptive sampler is worthwhile \emph{provided the signal
      quality justifies its extra parameters}. For low-SNR recordings the
      simpler CNN offers a better accuracy-per-FLOP trade-off.

\item[\textbf{RQ-C}]
      Long, random motion segments are crucial: when the model is denied EMG,
      validation error is expected to raise up to \textbf{7mm} more with random motion.
      We therefore recommend that future datasets favour unscripted activities
      to ensure that any measured gain truly comes from neuromuscular cues.

\end{description}

\section{Limitations and Future Work}

\begin{itemize}
  \item \textbf{Analogue noise floor}.
        The Seeeduino`s 12-bit ADC and hobby-grade Bio-Circuit amplifiers are
        the dominant noise source. A drop-in board based on a
        16-bit ADS1299 (or comparable TI/Biopac front-end) would raise
        dynamic range by \(\sim\)\SI{18}{dB}.

  \item \textbf{Sampling throughput}.
        Python capture is capped at \SI{32}{fps}. Re-implementing the
        triangulation pipeline in C++ or Rust (with OpenCV bindings) should
        push the full stack to \textbf{60 fps real time}, bringing optical
        frame rate in line with the webcams' native output.

  \item \textbf{Occlusion-aware triangulation}.
        Currently all 21 landmarks are triangulated with equal weight. Future
        work could score each 2-D detection by per-camera visibility and
        confidence, then solve a weighted least-squares system that
        automatically down-weights occluded or low-confidence points.

  \item \textbf{Global calibration}.
        A bundle-adjust refinement over all camera pairs—rather than the
        present stereo-pivot shortcut—should shave residual error and extend
        depth accuracy.

  \item \textbf{Model compression / STS featurizer overhaul}.
        The STS variant is larger and slower than CNN. Pruning, quantisation,
        or knowledge-distillation could reduce latency without sacrificing the
        benefits of adaptive sampling. At the same time we can try to constraint
        the DOFs of the patterns -- stick to family of spike functions to be used.
        Allow for slice to be matched with a portion of patterns instead of
        matching with all og the presented ones.

  \item \textbf{Learnable temporal windows}.
        An encouraging avenue is to make the slice widths and statistical
        window sizes \emph{themselves} learnable, allowing the network to tune
        its temporal receptive field jointly with spatial attention.

  \item \textbf{Extended kinematic expressiveness}.
        Our IK layer assumes uniform bone lengths and a flat palm. Future work
        could personalise bone lengths per user, add palm arching, wrist
        pronation/supination, and even non-rigid soft-tissue deformation,
        allowing the network to model a wider range of morphologies and motion
        styles.

\end{itemize}

\section{Final Remarks}

This work shows that high-quality, multi-modal hand-pose datasets can be
captured on a student budget and that adaptive EMG sampling pays off once the
signal-to-noise ratio crosses a modest threshold.
We hope both the open hardware plans and the released software stack will lower
the barrier for future studies and encourage a shift toward longer, more
naturalistic recordings -- a prerequisite for reliable prosthetic control and
intuitive AR/VR interaction.