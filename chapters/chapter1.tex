\chapter{Introduction}
\label{chap:intro}
\chaptermark{Introduction}

\section{Background}
Surface Electromyography (sEMG) is a non-invasive technique that records the electrical activity of muscles during contraction, providing valuable insights into muscle function. It is widely applied in fields such as prosthetics, rehabilitation, and human-machine interaction, where accurate decoding of muscle activity is crucial for controlling devices or understanding motor impairments \cite{farago2022review}. Surface EMG signals, however, present challenges due to their complex, nonlinear, and non-stationary nature, as well as the presence of noise and variability across individuals and tasks \cite{zia2018multiday, parajuli2019real}.

Recent advancements in machine learning, particularly deep learning, have revolutionized sEMG analysis by enabling automatic feature extraction and improved performance \cite{zhang2023lstm,ameri2019regression}. Convolutional Neural Networks (CNNs) have shown promise in decoding spatial and temporal features directly from raw sEMG data, while attention mechanisms provide interpretability and selective focus on relevant signal components \cite{lee2022explainable}. Despite these advancements, a comprehensive and robust solution for continuous hand motion estimation remains elusive.

\section{Novelty and Research Gap}
Despite significant progress, continuous hand motion estimation using sEMG remains a complex challenge due to signal variability, noise, and the need for high temporal resolution to support real-time applications \cite{farago2022review, zia2018multiday}. Kolmogorov-Arnold Networks (KANs), renowned for their capacity to approximate high-dimensional and nonlinear functions, offer a promising yet underexplored approach for addressing these complexities \cite{liu2024kan}.

This research addresses the gap by proposing a unified framework that integrates KANs, CNNs, and attention mechanisms to achieve accurate and robust continuous hand motion estimation. The novelty lies in leveraging the complementary strengths of these methods — KANs for universal approximation, CNNs for automatic feature extraction, and attention mechanisms for adaptive focus.

\section{Purpose and Objectives}
The primary objective of this research is to develop a novel approach for continuous hand motion estimation using sEMG signals. The specific objectives are:
\begin{enumerate}
    \item To explore the application of Kolmogorov-Arnold Networks (KANs) for modeling the nonlinear and dynamic nature of sEMG signals.
    \item To integrate CNNs and attention mechanisms into the framework to enhance feature extraction and focus on relevant signal patterns.
    \item To evaluate the proposed approach on benchmark sEMG datasets for accuracy, robustness, and real-time applicability.
\end{enumerate}

\section{Research Questions}
This study aims to answer the following research questions:
\begin{itemize}
    \item How can Kolmogorov-Arnold Networks (KANs) be applied to sEMG signal analysis for continuous hand motion estimation?
    \item What advantages does a unified approach combining KANs, CNNs, and attention mechanisms offer over existing methods?
    \item Can the proposed framework achieve real-time applicability while maintaining accuracy and robustness?
\end{itemize}

\section{Significance and Implications}
This research has significant implications for fields such as prosthetics and rehabilitation, where precise and reliable decoding of muscle activity is critical. By addressing the limitations of existing methods, the proposed framework could enhance the usability and functionality of sEMG-based systems, enabling more natural and intuitive control of devices. Furthermore, the findings could inform future research on integrating advanced neural networks for biomedical signal analysis.

\section{Structure of the Paper}
The remainder of this thesis is organized as follows:
\begin{itemize}
    \item \textbf{Chapter 2}: Provides a comprehensive review of related work in sEMG analysis, highlighting the challenges and existing solutions.
    \item \textbf{Chapter 3}: Describes the methodology, including the design and integration of KANs, CNNs, and attention mechanisms.
    \item \textbf{Chapter 4}: Presents the experimental setup and results, evaluating the performance of the proposed framework.
    \item \textbf{Chapter 5}: Discusses the implications of the findings and concludes with directions for future research.
    \item \textbf{Chapter 6}: Concludes the study by summarizing key insights, highlighting contributions, and discussing limitations and potential avenues for future research.
\end{itemize}