\chapter{Introduction}
\label{chap:intro}
\chaptermark{Introduction}

\section{Background and Motivation}

Surface electromyography (sEMG) has emerged as a key technique for interpreting human muscular activity, with wide applications in prosthetics, human-computer interaction, and rehabilitation \cite{zheng2022surface, simao2019review}. However, accurate hand pose estimation from sEMG signals remains a challenging task due to the complexity of muscular activation patterns and the variability in signal interpretation across users and tasks \cite{farago2022review, zia2018multiday}.

Existing datasets used for sEMG-based hand pose estimation often emphasize inter-subject generalization, focusing on short and structured sessions with a limited number of predefined gestures \cite{salter2024emg2pose}. While this approach suits some applications, it fails to explore the boundary of intra-subject generalization across a broader set of complex, continuous hand motions.

A significant limitation lies in the available multimodal dataset capturing systems that combine EMG with 3D hand tracking. High-end solutions, such as the EMG2Pose system \cite{salter2024emg2pose}, rely on 26 professional-grade motion capture cameras, resulting in cost and complexity levels that are prohibitive for most research labs. At the other end of the spectrum, low-cost setups often fail to meet essential requirements — for example, they may operate at suboptimal sampling rates (e.g., below 2kHz) or lack synchronized, continuous hand tracking altogether \cite{nasri2020semg}.

As a result, there is a lack of accessible and capable systems that provide both high-quality sEMG data (at least 6 channels, 12bit $\geq$2\,kHz) and accurate, real-time hand tracking at 30fps or higher \cite{graf2023combining}. These limitations hinder the ability to collect long-duration, high-fidelity datasets suitable for evaluating detailed hand pose estimation models under realistic conditions.

This work addresses these limitations by proposing a cost-effective, synchronized, and high-quality multimodal data acquisition system and a novel modeling technique tailored for spatiotemporally-rich sEMG signals.

\section{Novelty and Research Gap}

The primary contribution of this thesis is a new approach to modeling sEMG signals called \textit{Spatiotemporal Sampling}. Unlike traditional models that use fixed windows and treat all channels equally, our method dynamically learns both the temporal window and the spatial activation pattern across EMG channels. This allows the model to compute an inference output based on the similarity between a learned spatiotemporal pattern and the signal content within a contextually learned window. The term \textit{sampling} in this context refers to the adaptive extraction of signal segments in both space (across channels) and time.

Complementing this model, we have developed a novel data acquisition pipeline that integrates:
\begin{itemize}
    \item Continuous 30fps tracking of finger joint angles using triangulated projections from MediaPipe Hands across multiple cameras.
    \item Real-time inverse kinematics computation for full 3D hand pose reconstruction.
    \item A custom-built, cost-effective EMG acquisition system with six channels sampled at 2kHz.
\end{itemize}

Together, these contributions address a crucial gap in current research: the lack of systems that allow testing hand pose estimation models under longer, richer intra-user sessions with diverse motion sets. To demonstrate the broad applicability of the proposed model, it was evaluated both on this newly collected dataset and on an existing public dataset.

\section{Purpose and Objectives}

The purpose of this thesis is to explore and validate a novel modeling and data acquisition framework for sEMG-based hand pose estimation that enables:
\begin{itemize}
    \item Learning flexible, informative spatiotemporal features from multichannel sEMG signals.
    \item Capturing hand motion data aligned with sEMG activity through a cost-efficient system.
    \item Investigating the limits of intra-user generalization over prolonged and varied motion sequences.
    \item Benchmarking the proposed model on both existing datasets and newly collected data.
\end{itemize}

To achieve this, the specific objectives include:
\begin{enumerate}
    \item Designing and implementing the Spatiotemporal Sampling model.
    \item Developing a synchronized multimodal hand tracking and EMG capture system.
    \item Conducting long-session experiments to collect a novel dataset with diverse motion patterns.
    \item Evaluating the model on both the new dataset and a public benchmark dataset.
\end{enumerate}

\section{Significance and Implications}

This thesis presents a new approach for modeling temporal and spatial characteristics of sEMG signals for hand pose estimation. The proposed method uses spatiotemporal sampling to extract a learned pattern from multichannel EMG input and measure its similarity to the signal within a learned temporal window.

In parallel, the introduced dataset capture system bridges a critical gap between cost, usability, and data fidelity. It enables researchers and developers to collect high-quality EMG and motion data without relying on prohibitively expensive commercial setups \cite{quivira2018translating}.

Testing the proposed model on both a public dataset and the custom-built multimodal dataset provides an initial evaluation of its performance and applicability under different data collection conditions. These developments contribute to the broader research efforts in wearable systems, personalized rehabilitation, and human-machine interaction.
