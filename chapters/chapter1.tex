\chapter{Introduction}
\label{chap:intro}
\chaptermark{Introduction}

\section{Background and Motivation}

Surface electromyography (sEMG) is now one of the top methods we use to study muscle activity, picking up the low-level electrical signals that muscle fibers generate through electrodes placed on the skin. It is now one of the main ways we read the activity of human muscles, and it shows up in lots of areas like prosthetic hands, human-computer interaction, and rehab therapy \cite{zheng2022surface, simao2019review}. Since the sensors rest directly on the skin and can capture the very little details of muscle electro activity that can be interpreted as a motion counterpart [TODO: supporting ref here with emg signal formulas], we have great intuition that by analyzing the signal we can decode the muscle state and contraction very accurately. The importance of adopting sEMG-based approaches is furthermore motivated by the freedom that it will offer. Unlike optical-based hand motion tracking, sEMG offers to work with less requirements: no lightning is needed, no occlusion problem is presented - hence sEMG overall can capture more variable hand motion without severe tracking interruptions.

At the same time, accurately estimating hand pose from sEMG signal remains a complex challenge due to the variability of the pure signal parameters - producing highly individualized muscle activation patterns, that is furthermore worsened by the environment noise \cite{farago2022review, zia2018multiday}. Therefore, finding reliable and generalizable approaches to decode sEMG into hand motion is a very non-trivial task, that was widely explored starting from classical approaches relying on hand-build features to modern LSTM/Attention techniques [TODO: put ref on SVM approaches and other classical ones], although is still an actively developing area of research where no universal method found. This ongoing exploration highlights the importance of investigating new modeling strategies that can better capture the spatiotemporal dynamics of muscle activity.

Existing datasets used for sEMG-based hand pose tracking and estimation emphasize inter-subject generalization, focusing on short and structured emg to hand motion recordings of fixed repeatable motions and with a limited number of predefined gestures \cite{salter2024emg2pose}. While this approach suits some applications, it fails to explore the boundary of intra-subject generalization across a broader set of complex, continuous hand motions at first, but also makes the hand motion patterns in the train data to be too predictable so that models tend to overfit hence showing poor generalization.

For one that needs his own dataset a significant obstacle is observed on the field of available multimodal capturing setups that combine EMG with 3D hand tracking. High-end solutions, such as the EMG2Pose system \cite{salter2024emg2pose}, rely on 26 professional-grade motion capture cameras and a proprietary sEMG band, resulting in excessive costs and complexity levels. At the other end of the spectrum, low-cost setups often fail to meet essential requirements — for example, they may operate at suboptimal sampling rates (e.g., below 1kHz) or lack synchronized, continuous hand tracking altogether at all, by predicting only hand pose or motion classes \cite{nasri2020semg}.

Consequently, there is a shortage of user-friendly and effective systems that can simultaneously generate high-quality sEMG data and accurately track hand movements in real time \cite{graf2023combining}. These shortcomings prevent the collection of extensive and precise datasets that can be used to assess sophisticated hand pose estimation models in practical settings.

\section{Novelty and Research Gap}

To overcome the obstacles outlined above, this research proposes two major contributions: a novel method for interpreting sEMG signals and a cost-efficient system for collecting multimodal data, specifically tailored for capturing real-world, unrestricted hand movements, with the goal of achieving consistent results across a diverse array of intricate and unrestricted hand motions.

To address the issue of disparity in modeling approaches that heavily rely on computationally intensive architectures and lack precise control over temporal synchronization, this work introduces a novel technique called \textit{Spatiotemporal Sampling}. While conventional models — such as CNN where suited for dynamic pattern recognition, they often do so at the cost of higher computational complexity and less interpretable temporal structure. In contrast, the proposed approach is utilizing the static nature of the important parts of the signal - it dynamically learns both the optimal temporal context and the most relevant spatial activation patterns across channels in a lightweight and efficient manner hence freezing the window on the one important window and wasting no resources on the full scan. The model computes predictions by comparing the current input to a learned spatiotemporal signature, allowing for faster training, reduced inference cost, and more compact parameterization. The term \textit{sampling} in this context refers to the selective, learned extraction of signal segments in both time and space, enabling the model to focus on the most informative patterns while maintaining robustness and adaptability to signal variability.

\begin{figure}[H]
    \centering
    \includegraphics[width=1.0\textwidth]{sps.drawio.png}
    \caption{Spatiotemporal Sampling.}
    \label{fig:sps}
\end{figure}

The concept of spatiotemporal sampling was motivated by the physiological phenomenon known as electromechanical delay (EMD) — the latency between muscle activation and observable movement \cite{ngeo2014continuous}. In this work, we operate under the assumption that the \textit{Observed EMD} can be expressed as:
\[
\text{Observed EMD} = \text{Natural EMD} + \text{Pose Input Lag} - \text{Signal Input Lag}
\]
where each component is treated as a near-constant, assuming the variances introduced are relatively small and stable. In particular, achieving consistent input lags requires the use of precise hardware clocks, while the absence of system overloads ensures no timing drain during data acquisition. Natural EMD is considered physiologically stable and is expected to drift so slowly over time that it remains practical to compensate for it through periodic recalibration. This formulation implies a consistent and reliable temporal alignment between muscle activation and resulting motion. Therefore, if a spatiotemporal pattern is detected within a learned temporal slice, it can be reasonably assumed that a corresponding physical movement is occurring at the same moment in real life. This justifies the model's use of focused, learnable windows instead of uniformly sliding ones.

Complementing this novel signal featurizer is the development of a new multimodal data acquisition pipeline perfectly balanced between the cost and quality by removing all the redundant parts of the higher-cost setups at the same time incorporating sophisticated heuristics for hand-specific tracking. The proposed system includes:
\begin{itemize}
    \item Continuous 30fps tracking of finger joint angles using triangulated 2D projections from MediaPipe Hands across multiple synchronized RGB cameras.
    \item Real-time inverse kinematics for reconstructing full 3D hand pose from tracked keypoints.
    \item A custom-built, low-cost EMG acquisition system with six channels sampled at up to 2kHz with 12-bit resolution, supporting synchronized multimodal capture.
\end{itemize}

Together, these contributions fill a critical research gap. On the one hand, they provide a practical and scalable way to collect high-quality, long-duration sEMG and hand motion data under realistic conditions. On the other, they offer a new modeling paradigm that is capable of leveraging the rich spatiotemporal structure in such data. Notably, the new dataset is specifically designed to enable the study of intra-subject generalization over complex and continuous hand motions — an area underexplored in prior work that predominantly focuses on predefined gesture classification in short trials \cite{salter2024emg2pose}.

To validate our approach, we will run experiments on both the newly collected dataset and a well-known public benchmark. These tests will reveal whether spatiotemporal sampling improves performance in practical, continuous settings. Alongside the method itself, we built an affordable data-collection system and dataset whose signal quality rivals that of far more expensive, high-end setups.

\section{Purpose and Objectives}

The purpose of this thesis is to explore and validate a novel modeling and data acquisition framework for sEMG-based hand pose estimation that enables:
\begin{itemize}
    \item Learning flexible, informative spatiotemporal features from multichannel sEMG signals.
    \item Capturing hand motion data aligned with sEMG activity through a cost-efficient system.
    \item Investigating the limits of intra-user generalization over prolonged and varied motion sequences.
    \item Benchmarking the proposed model on both existing datasets and newly collected data.
\end{itemize}

To achieve this, the specific objectives include:
\begin{enumerate}
    \item Designing and implementing the Spatiotemporal Sampling model.
    \item Developing a synchronized multimodal hand tracking and EMG capture system.
    \item Conducting long-session experiments to collect a novel dataset with diverse motion patterns.
    \item Evaluating the model on both the new dataset and a public benchmark dataset.
\end{enumerate}

\section{Significance and Implications}

This thesis presents a new approach for modeling temporal and spatial characteristics of sEMG signals for hand pose estimation. The proposed method uses spatiotemporal sampling to extract a learned pattern from multichannel EMG input and measure its similarity to the signal within a learned temporal window.

In parallel, the introduced dataset capture system bridges a critical gap between cost, usability, and data fidelity. It enables researchers and developers to collect high-quality EMG and motion data without relying on prohibitively expensive commercial setups \cite{quivira2018translating}.

Testing the proposed model on both a public dataset and the custom-built multimodal dataset provides an initial evaluation of its performance and applicability under different data collection conditions. These developments contribute to the broader research efforts in wearable systems, personalized rehabilitation, and human-machine interaction.
