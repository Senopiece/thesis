\chapter{Methodology}
\label{chap:met}
\chaptermark{Methodology}

This chapter documents all engineering steps taken to build and exploit our
custom multimodal sEMG + hand-pose dataset. All code is MIT-licensed; the data
are provided free of charge with no usage restrictions.

\section{Why a New Dataset?}

\begin{itemize}
  \item \textbf{Hardware parity.} EMG2Pose uses 26 OptiTrack cameras and a
        proprietary CTRL-labs wristband—out of reach for a \$1 000 budget.
  \item \textbf{Temporal realism.} Public corpora favour <10 s gesture clips;
        we require minutes-long, unscripted motion.
  \item \textbf{Storage efficiency.} Using a custom binary format, we
        reduce EMG2Pose`s session files to $\approx2.5\,\%$ of it's size, with no loss of
        information.
\end{itemize}

Further instructions are a top-level overview of the overall data acquisition pipeline, for detailed instructions refer to the contents of \href{https://github.com/Senopiece/webcam_hand_triangulation/tree/v1.0}{\texttt{the repository}}. [TODO: сноска с сылкой внизу]

\section{Four-Camera Hand-Pose Subsystem}

\subsection{Geometry and Calibration}

Four webcams (1280x800, 60 fps) are mounted around a
\SI{1.2}{\metre}$^{3}$ workspace—three frontal views aimed at the palm and one
dorsal view.  Accurate multi-view geometry is obtained with a \textit{radon
checkerboard}: an \emph{asymmetric} grid printed on \textbf{both} faces of an
A4 sheet, with extra anchor dots so the two patterns coincide perfectly when
held up to the light.  Thanks to this dual-face design every camera can see a
valid checkerboard in the \emph{same} shot, eliminating the 180° ambiguity of
single-face boards.

\begin{enumerate}[label=\arabic*.]
  \item \textbf{Per-camera intrinsics}\\
        \verb|cv::findChessboardCorners| detects the pattern in live frames
        streaming from each webcam.  Once every camera has logged
        $\ge12$ detections, \verb|cv::calibrateCamera| estimates the intrinsic
        matrix $K$ and distortion coefficients~$D$ individually
        (RMS reprojection error < \SI{0.3}{px}).

  \item \textbf{Pair-wise stereo calibration}\\
        One camera is designated the \emph{pivot}; by default the first index
        found in \texttt{cameras.def.json5}.  For each remaining camera, the
        script runs \verb|cv::stereoCalibrate| against the pivot while
        \emph{fixing} the intrinsics from step 1.  
        This yields a rotation $R$ and translation $T$ that map points from the
        pivot’s frame to the partner camera.

  \item \textbf{Write-out and visual check}\\
        All intrinsic blocks and $R\!T$ pairs are dumped to
        \texttt{cameras.calib.json5}.  An optional preview window shows the
        rectified chessboard overlay on every camera so the operator can
        visually confirm alignment before recording.
\end{enumerate}

\noindent
The current implementation relies on direct \emph{pair-wise} overlaps with the
pivot camera; a note in the code highlights a future upgrade to bundle-adjust
\emph{all} pairs for even tighter global consistency.  Nevertheless, the
stereo-against-pivot approach already provides sub-millimetre triangulation
accuracy at a working distance of \SI{50}{\centi\metre}, adequate for EMG-to-pose
learning.

\subsection{Landmark Triangulation \& Real-Time IK}
MediaPipe Hands detects 21 landmarks per retained frame (32 fps after
alignment). Multi-view triangulation yields 3-D key-points. A hand-crafted,
analytic inverse-kinematics solver enforces bone lengths and joint limits and works in real time.

\section{sEMG Acquisition Hardware}

\subsection{Electrode Montage}
Twelve Ag/AgCl electrodes form four bipolar channels + reference:

\begin{enumerate}[label=\alph*]
  \item \textbf{FDS} - \emph{Flexor Digitorum Superficialis}: the main finger-flexor mass on the palmar-ulnar side; activates when the fingers close.
  \item \textbf{FCR} - \emph{Flexor Carpi Radialis}: wrist flexor on the palmar-radial side; contributes to wrist flexion and radial deviation.
  \item \textbf{EDC} - \emph{Extensor Digitorum Communis}: dorsal-radial finger extensors; drives finger extension and assists in wrist extension.
  \item \textbf{ECU} - \emph{Extensor Carpi Ulnaris}: dorsal-ulnar wrist extensor; extends the wrist and produces ulnar deviation.
\end{enumerate}

\subsection{Electronics}
\begin{itemize}
  \item Six Bioelectronic-Circuit amps connected to Seeeduino Xiao (12-bit ADC).
  \item Sampling: \SI{2048}{Hz} x 4 channels.
  \item Streaming: USB 2.0 CDC, sustained $\approx \SI{2}{\mega\bit\per\second}$.
  \item No analogue or digital filtering at capture; any filtering is delegated
        to downstream models.
\end{itemize}

\section{Dataset Infrastructure}
\begin{itemize}
    \item \textbf{Real-time I/O optimisations}.  
          The capture software is build around a multithreaded
          worker-pool design. Careful use of mutexes and ring buffers resolved
          the synchronisation hazards that arise when high-rate EMG and video
          streams are written concurrently to disk.
    \item \textbf{Compact session format}.  
          Each recording session is stored as a single ZIP archive containing
          \textbf{YAML metadata} plus proprietary binary segments for EMG and
          pose.  Compared with an HDF5 dump of the same content, the archive
          shrinks disk usage by roughly \textbf{40x}. The public EMG2Pose
          corpus was re-encoded into this layout so that all datasets share one
          structure.
\end{itemize}

\subsection{Synchronized Recording}

\begin{itemize}
  \item Cameras run at 60 fps; every second frame is kept 32 fps pose stream.
  \item Exactly 64 EMG samples (\SI{31.3}{ms}) align to each pose frame. The
        logger polls the \emph{latest} camera image at each 64-sample boundary.
  \item Timestamp jitter is negligible; no interpolation is
        applied for alignment.
\end{itemize}

\subsection{Session Format}

\paragraph{Layout}

\begin{verbatim}
session.zip/
  metadata.yml
  recordings/
    1/segments/1 2 …
    2/segments/1 2 …
\end{verbatim}

\paragraph{Segment Binary Layout}

Each \texttt{segments/N} file contains a sequence of tuples:

\[
\bigl[
  [\underbrace{20\! \times\! \text{float32}}_{\text{pose}},
   \underbrace{64\!\times\!4}_{\text{EMG}}\text{ float32}],
  \dots,
  \underbrace{20\! \times\! \text{float32}}_{\text{sigma frame}}
\bigr]
\]

The final 20-value \emph{sigma frame} holds the last pose sample that is not coupled with EMG.

\subsection{Quality Control}

The capture application shows

\begin{enumerate}[label=\alph*]
    \item \textbf{EMG oscilloscope}.  
          Scrolling plots of each channel allow the operator to verify signal amplitude, baseline drift, and electrode contact before and during recording.
    \item \textbf{Session control panel}.  
          A GUI window exposes start/stop buttons and a built-in timer that counts elapsed recording time, reducing annotation errors.
    \item \textbf{Interactive 3-D hand preview}.  
          A interactive (rotate/zoom) viewer renders the triangulated skeleton so occlusions or tracking drop-outs can be spotted immediately.
\end{enumerate}

Immediate feedback helps catch loose electrodes or camera drop-outs.

\subsection{Interpolation \& Resampling Strategy}

\paragraph{Pose formats.}
The loader supports two 20-angle representations:

\begin{itemize}
  \item \textbf{UmeTrack} — the 20-DOF joint-angle vector used in the original
        EMG2Pose release.  For strict comparability with that work, we keep the
        original \emph{linear} interpolation.
  \item \textbf{AnatomicAngles} — our own 20-angle layout, reordered to match
        surface-EMG anatomy.  Here we adopt \textbf{Akima splines}, which give
        smoother, overshoot-free trajectories well suited to biomechanical
        data.
\end{itemize}

\paragraph{Kernel benchmark.}
Five interpolation techniques were timed on a 250$\to$1000-frame up-sample test:

\begin{center}\small
\begin{tabular}{@{}lcc@{}}
\toprule
Scheme            & Time (ms) & Visual verdict \\ \midrule
Linear            & 1.0  & robotic, piecewise-constant velocity \\
\textbf{Akima}    & 6.4  & smooth, no overshoot (chosen) \\
Monotone cubic    & 8.3  & smooth but noticeably slower \\
Cubic spline      & 6.7  & severe overshoot at peaks \\
B-spline          & 9.3  & same overshoot issue, slowest \\ \bottomrule
\end{tabular}
\end{center}

Akima offers the best trade-off between speed and anatomical plausibility.

\paragraph{Loader-time resampling.}
Raw pose is stored exactly as captured at 32 fps.  
When a model requests a different label rate (\{16, 32, 64, 128\} fps),
the dataset class resamples on the fly:

\begin{itemize}
  \item \textbf{UmeTrack} $\to$ linear interpolation,
  \item \textbf{AnatomicAngles} $\to$ Akima interpolation.
\end{itemize}

No interpolation is applied during live capture or camera-EMG alignment;
resampling is purely an offline, loader-side convenience.




TODO: about dataset split on the model side, model description
