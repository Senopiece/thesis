\chapter{Methodology}
\label{chap:met}

The overall workflow is organized into two major phases: first, building our own multimodal dataset; second, training and benchmarking several EMG-to-pose models on both public data and the new recordings.

\section{Phase 1 — In-House Dataset Acquisition}

\begin{enumerate}[label=\textbf{1.\arabic*.}, wide=0pt, leftmargin=*]
    \item \textbf{Design a Hand-Tracking Subsystem}\\
          Select — and if necessary assemble — a vision pipeline capable of estimating 3-D hand pose at interactive frame rates.  The exact hardware layout is flexible: options range from a single depth sensor to a small multi-camera rig running marker-less tracking.  The key requirements are decent spatial coverage, low latency, and an open interface for downstream synchronization.
    \item \textbf{Develop an sEMG Capture Unit and Software}\\
          Build or adapt a wearable EMG device that samples multiple forearm channels at a sufficiently high rate and resolution.  Provide accompanying firmware and a host application so that raw signals, timestamps, and metadata can be streamed or logged in real time.
    \item \textbf{Synchronize and Integrate Both Streams}\\
          Establish a common time base (either hardware triggers or software time-stamps) so that each EMG sample aligns with its corresponding hand-pose frame.  Package the paired readings into a single container format along with subject notes, electrode maps, and camera parameters.
    \item \textbf{Record Continuous, Natural Sessions}\\
          Use the integrated setup to collect long, free-form recordings that capture everyday hand movements rather than short, scripted gestures.  These sessions constitute the new dataset used later for training and evaluation.
\end{enumerate}

\section{Phase 2 — Model Design, Training, and Evaluation}

\begin{enumerate}[label=\textbf{2.\arabic*.}, wide=0pt, leftmargin=*]

    \item \textbf{Build Two Featuriser Variants}\\
          \begin{enumerate}[label*=\arabic*.]
              \item \emph{CNN Baseline} — a compact 1-D convolutional front-end followed by an LSTM or MLP head.  
              \item \emph{STS Model} — the proposed \textit{Spatiotemporal Sampling} block replaces the CNN front-end, learning per-instance electrode and time-window weights.  
          \end{enumerate}

    \item \textbf{Hyperparameter Search Across Datasets}\\
          \begin{enumerate}[label*=\arabic*.]
              \item Define a common search space (learning rate, window length, channel dropout, optimiser, etc.).  
              \item Run Bayesian or population-based search separately on the public EMG2Pose split and on the in-house dataset, logging the best configuration for each featuriser.  
              \item Record validation metrics at each trial so that search behaviour is reproducible.
          \end{enumerate}

    \item \textbf{Select Best-Performing Hyperparameters}\\
          Using the best hyperparameters obtained on the previous step, deeper explore it's fitness amongst both the public EMG2Pose split and on the in-house dataset.

    \item \textbf{Comparative Analysis}\\
          \begin{itemize}
              \item \textit{Featuriser Effect} — contrast best CNN versus best STS on the same dataset to quantify gains from adaptive sampling.  
              \item \textit{Dataset Effect} — contrast models trained on EMG2Pose versus those trained on the in-house data to see how well each corpus supports generalisation.  
          \end{itemize}

\end{enumerate}